% ----------------------------------------------
% Author: Felipe Alfonso González
% License: MIT
%
%%%%%%%%%%%%%%%%%%%%%%%%%%%%%%%%%%%%%%%%%

%----------------------------------------------------------------------------------------
%	PACKAGES AND OTHER DOCUMENT CONFIGURATIONS
%----------------------------------------------------------------------------------------

\documentclass[
    a4paper, % Paper size, use either a4paper or letterpaper
    10pt, % Default font size, can also use 11pt or 12pt, although this is not recommended
    unnumberedsections, % Comment to enable section numbering
    twoside, % Two side traditional mode where headers and footers change between odd and even pages, comment this option to make them fixed
]{LTJournalArticle}

\usepackage{amsmath}

\usepackage{hyperref}

\addbibresource{biblio.bib} % BibLaTeX bibliography file

\runninghead{} % Leave this command empty for no running head

\footertext{} % Leave this command empty for no footer text

\setcounter{page}{1} % The page number of the first page, set this to a higher number if the article is to be part of an issue or larger work

\usepackage{amsmath} % Add this line to load the amsmath package

%----------------------------------------------------------------------------------------
%	TITLE SECTION
%----------------------------------------------------------------------------------------

\title{Reevaluando la Eficacia de los Desafíos de Código: Desmitificando la Idea de un Ingeniero como una IA Omnisciente}

% Authors are listed in a comma-separated list with superscript numbers indicating affiliations
% \thanks{} is used for any text that should be placed in a footnote on the first page, such as the corresponding author's email, journal acceptance dates, a copyright/license notice, keywords, etc
\author{%
    Felipe Alfonso González\textsuperscript{1}\thanks{Corresponding author: \href{mailto:f.alfonso@res-ear.ch}{f.alfonso@res-ear.ch}}\\
    \textit{ Computer Science Engineer}\\
    \footnotesize Institute of Arts and Communication Sciences (IACC), Chile\\[-6pt]
    \footnotesize Candidate for Master in Big Data, ENEB / Isabel I University\\[-6pt]
    \footnotesize\href{mailto:f.alfonso@res-ear.ch}{f.alfonso@res-ear.ch} - 
    \href{https://glzengrg.com}{glzengrg.com} - 
    Twitter: \href{https://twitter.com/felipealfonsog}{@felipealfonsog} - 
    LinkedIn: \href{https://linkedin.com/in/felipealfonsog}{felipealfonsog}\\
    \scriptsize This manuscript has been authored using the typesetting system \LaTeX{}. \\[-6pt]
    \scriptsize This manuscript is released under the BSD 3-clause License. \\
}


% Full-width abstract
\renewcommand{\maketitlehookd}{%
    \begin{abstract}
        \noindent Code challenges have become a popular method for assessing technical skills in software engineering hiring processes. However, this paper aims to challenge the efficacy of such challenges by examining their limitations and questioning the misconception that engineers should possess an all-encompassing knowledge akin to artificial intelligence. This paper advocates for a more holistic assessment approach that takes into account problem-solving skills, teamwork, and adaptability. The proposed approach encourages a shift from testing rote memorization to evaluating an engineer's ability to learn, collaborate, and innovate. This paper presents a comprehensive critique of code challenges and proposes a paradigm shift in the evaluation of engineering talent.
        
    \end{abstract}
}

%----------------------------------------------------------------------------------------



\begin{document}

\maketitle % Output the title section

%----------------------------------------------------------------------------------------
%    ARTICLE CONTENTS
%----------------------------------------------------------------------------------------

\section{Introduction}

In today's rapidly evolving software engineering landscape, the assessment of technical skills for prospective engineers has gained considerable importance. Code challenges, a widely employed approach, have become the de facto standard for evaluating a candidate's coding capabilities within a constrained time frame. However, the efficacy of code challenges as a sole indicator of engineering competence warrants closer examination. This paper aims to critically evaluate the limitations of code challenges and advocate for a more comprehensive assessment approach that reflects the multifaceted nature of modern software engineering \cite{Smith2020,Johnson2019,Brown2021,Garcia2022}.

\subsection{Limitations of Code Challenges}

Code challenges, while providing valuable insights into an engineer's problem-solving prowess, suffer from inherent limitations. By design, code challenges prioritize algorithmic efficiency and rapid problem-solving over other crucial attributes such as creativity, adaptability, and teamwork. This singular focus can lead to a skewed evaluation of a candidate's abilities, potentially overlooking the soft skills that are integral to successful software development.



%%%%%%% new stuff to be improved 

A pair programming is a really bad idea to try to see the potential in a possible candidate, it doesn't adjust the needs either way, because you are going to interact with someone you don't know, someone with a completely unknown attitude and different abilities to complete a task, therefore as a candidate, there will be chances that you interview will be a complete failure, a complete catastrophe, it's stupid, even if you have been preparing yourself for months something unexpected could happen. 

%Insert stats about this a math stat




%——————-


HISTORY OF A LONG STORY TO UNDERSTAND DEVELOPERS 



ITS PRETTY UTIPIC TO BELIVE THAT SOMEONE WILL BECOME PRACTIXALLY A AI IN A TECHNICAL INTERVIEW. it's a complete non sense. 


IN THE FUTURE WE MIGHT SEE A CHANCE TO SEE AN EVEN NEW BRANCH IN ENGKNEERng, like prompt engineering with different approaches.


PSYCHOLOGICAL ASPECTS OF TECH INTERVIEWS 
——


POSSIBLE OPTIONS TO RESOLVE THIS ISSUES:
to expose candidates to the exploration and research in different topics in engineering and or computer science. testing something but considering the possibility for the candidate to research and explore solutions for a real world problem. 



TRUST : a factor in rhe process of finding the right csndisste 








\subsection{High Software Engineer Turnover: The Impact of Code Challenges}

Unfortunately, anyone involved in the field of computer science engineering is well aware of the stress and questionable efficacy associated with code challenges. It seems counterintuitive when one is asked to undergo a testing process where you're required to be in front of a group of people, often utilizing webcams, and using specialized software for pair programming.

The ubiquity of code challenges in the hiring process for software engineers has raised eyebrows and provoked discussions within the industry. The question arises: do these challenges truly assess the skills and qualities necessary for a successful software engineer, or are they a source of unnecessary anxiety and a potential deterrent for talented candidates?

The concept of evaluating technical abilities through problem-solving assessments is not inherently flawed. However, the implementation of such evaluations in the form of code challenges may not always align with the real-world scenarios that software engineers encounter. This discrepancy between the test environment and the actual work environment can lead to a skewed representation of an engineer's capabilities.

The high turnover rate among software engineers is a concern that industry leaders are actively trying to address. Factors contributing to this turnover include burnout, lack of growth opportunities, and misalignment between expectations and reality. The imposition of code challenges, especially those that may not accurately mirror the skills required for the job, could exacerbate these challenges and further contribute to the issue of high turnover.

\subsection{Statistical Insight}

To quantitatively highlight the impact of code challenges on software engineer turnover, we can formulate a statistical equation. Let $\text{Turnover}$ represent the high software engineer turnover, and $\text{Challenges}$ denote the prevalence of code challenges in hiring processes. We can posit the following equation:

\begin{equation}
    \text{Turnover} = k \times \text{Challenges}^2
\end{equation}

Where $k$ is a constant representing the amplification of turnover due to code challenges. This equation showcases how the use of code challenges can lead to a nonlinear increase in the software engineer turnover rate. The undue stress and potential misalignment of assessment with actual job requirements contribute to the rapid growth of turnover rates.



In conclusion, it's imperative to reassess the role of code challenges in the evaluation of software engineers. While the intention behind such assessments may be to gauge technical prowess, their application should be scrutinized to ensure they align with the complexities and demands of real-world software development. It's crucial to create an inclusive and accurate evaluation process that doesn't add unnecessary stress and hinder the retention of skilled engineers in the industry.


\subsection{Limitations of Traditional Technical Interviews}

Traditional technical interviews, while widely used in the software engineering hiring process, present notable limitations that call into question their effectiveness in accurately assessing candidates' expertise. In this section, we examine these limitations and shed light on the discrepancies between the intended purpose of such interviews and their actual outcomes.

\subsubsection*{Narrow Focus and Incomplete Assessment}

One glaring limitation of traditional technical interviews is their narrow focus on specific aspects, such as assessing candidates' knowledge of a particular programming language. This approach fails to provide a comprehensive evaluation of candidates' skills, overlooking critical attributes like problem-solving abilities, collaboration aptitude, and adaptability to evolving technologies. Consequently, the interview's outcomes may not accurately reflect a candidate's overall potential or fit within a dynamic software development environment.

\subsubsection*{Undue Stress and Performance Anxiety}

Technical interviews often create a high-stress environment for candidates, leading to performance anxiety that can hinder their true capabilities. The pressure of being scrutinized and evaluated on the spot can lead to nervousness and suboptimal performance, rendering the interview results less indicative of a candidate's actual skills and expertise. The artificial nature of these interviews fails to replicate real-world scenarios where problem-solving often occurs in a collaborative and less time-constrained environment.


%%%%%% new linea about health - mental health - state of mind 

Another thing that's really important is to understand how stressful it is to be updated and hence updated for any sort of interview where developers require e to kind of demonstrate that.. how? usually by projects or work done.. but it is not so simple.  


\subsubsection*{Misalignment with Industry Realities}

Another significant limitation is the misalignment between the structure of technical interviews and the complex realities of the software engineering industry. Modern software development emphasizes teamwork, communication, and iterative problem-solving, which traditional technical interviews tend to overlook. Isolating candidates from the collaborative context in which they would typically work fails to provide an accurate representation of their potential contributions to a development team.

%new lines 
%%%%%%%%%%%%%%%%%%%%%%

Frequently encountered online are a plethora of courses vying for attention, accompanied by enticing advertisements that extol the virtues of their offerings. Promises of rapid entry into the IT job market abound, painting an alluring picture that often veers far from reality. The stark truth is that the journey towards becoming proficient in IT is fraught with demanding challenges, far more arduous than many may anticipate. Beneath the veneer of ease, lies a reality that remains hidden from those who readily embrace these courses \cite{dweck2006mindset,duckworth2016grit}.

These promotions, however well-intentioned, frequently overlook the rigorous trials and tribulations that await aspiring learners. The trials, often relentless and formidable, stand as a formidable test of one’s resilience and determination. Enrollees, spurred on by these enticing prospects, seldom fathom the intensity and depth of commitment required to surmount these hurdles. Embarking on the journey without a genuine understanding of the inherent difficulties could lead to disillusionment and a potential crisis of confidence.

It is of paramount importance to underscore the significance of acquiring a solid educational foundation in engineering disciplines. The allure of expedited success and the immediate allure of lucrative job prospects should not eclipse the value of comprehensive learning and development. For those who tread the path without first dedicating themselves to the study of engineering principles, the result could be more than mere disappointment; it could lead to a pervasive feeling of unpreparedness and inadequacy.

In the pursuit of a career in IT, the path is not a sprint but a marathon. True success is built on a sturdy foundation of knowledge, perseverance, and rigorous training. The misperceptions perpetuated by grandiose promises can culminate in a disheartening realization of the true challenges ahead. The prudent course of action is to approach the journey with open eyes and an earnest commitment to a disciplined education. In doing so, one can transcend the superficial allure of swift success and instead forge a solid and lasting trajectory towards a fulfilling career in the realm of IT \cite{tetlock2007illusion}.











\subsubsection*{Bias and Exclusion}

Traditional technical interviews may inadvertently introduce bias into the assessment process. The focus on specific technical skills can favor candidates from certain educational or experiential backgrounds while excluding qualified individuals who possess a broader skill set. This bias can lead to a lack of diversity within the engineering workforce and hinder the discovery of exceptional talents with non-traditional paths.

\subsubsection*{Ethical Concerns}

Ethical considerations surrounding technical interviews also deserve attention. Subjecting candidates to high-stakes evaluations without adequate consideration for their psychological well-being raises ethical questions. The potential negative impact on candidates' self-esteem, regardless of their performance, calls into question the ethical implications of relying solely on traditional technical interviews as a hiring tool.

In conclusion, while traditional technical interviews have been a staple in the software engineering hiring process, their limitations are significant. The narrow focus, stress-inducing nature, misalignment with industry practices, potential bias, and ethical concerns highlight the need for a more comprehensive and thoughtful approach to evaluating candidates' suitability for modern software development roles.



\subsection{Questioning the All-Knowing Engineer Paradigm}

The prevailing notion of an engineer as an all-knowing entity well-versed in every programming language, framework, and technology is unrealistic and counterproductive. In an era of continuous technological advancements, the expectation for engineers to possess an encyclopedic knowledge is untenable. Rather than striving for an impossible ideal, engineers should be encouraged to embrace a growth mindset, focusing on their capacity to learn, adapt, and collaborate effectively.

\subsection{The Need for a Holistic Assessment}

A holistic assessment approach acknowledges that software engineering is not solely about writing flawless code. Instead, it encompasses a spectrum of skills ranging from critical thinking to clear communication and from teamwork to innovation. A comprehensive evaluation process should account for these diverse dimensions, enabling organizations to identify candidates who are equipped to thrive in the dynamic and collaborative environment of modern software development.

%------------------------------------------------

\section{Methodologies}

To discern the effectiveness of code challenges and explore alternative assessment methods, a thorough analysis of existing literature, industry practices, and case studies was conducted. The goal was to elucidate the correlation between success in code challenges and long-term engineering accomplishments.

\subsection{Sample Sites \& Processing}

The study encompassed an extensive array of software engineering platforms, each hosting a variety of code challenges. Analyzing the outcomes of participants, including their subsequent contributions to open-source projects, was integral to evaluating the predictive value of code challenge success.

%------------------------------------------------

\section{The Fallacy of Comprehensive Expertise}

Engineering excellence is not synonymous with being a walking repository of technical minutiae. Rather, it involves cultivating foundational understanding and embracing the aptitude to navigate novel territories. The software engineering landscape is fluid, with languages and tools frequently evolving. Rote memorization pales in comparison to an engineer's capacity to comprehend fundamental principles and apply them innovatively.

\section{Incorporating Real-World Complexity}

Software development extends beyond isolated coding sprints. Collaborative team efforts, iterative design, and communication play pivotal roles. A comprehensive evaluation should emulate real-world scenarios, emphasizing how candidates contribute to a team's dynamic, provide constructive feedback, and navigate through diverse perspectives.

%------------------------------------------------

\section{Results and Implications}

The study revealed that while code challenges serve as an initial filter for technical skills, their predictive value diminishes concerning an engineer's collaborative aptitude, adaptability, and capacity to contribute to complex projects. Candidates excelling in code challenges do not inherently outperform their peers in real-world projects that demand adaptability and collective problem-solving.

\section{Learning from Failures}

The capacity to learn from setbacks is instrumental in software engineering. Rather than penalizing candidates for errors during assessments, it is more instructive to assess their approach to challenges, iteration on solutions, and incorporation of lessons learned. This adaptive resilience is a hallmark of a competent engineer.

%------------------------------------------------

\section{Redefining Engineering Excellence}

Engineering is an amalgamation of cognitive agility, problem-solving acumen, and interpersonal skills. An engineer's worth lies not in the mastery of a predefined set of skills, but in their ability to learn, collaborate, and innovate. By championing these traits, organizations foster a culture of resilience and evolution.

\section{Promoting a Paradigm Shift}

A transformative shift in assessment paradigms is overdue. Code challenges, while valuable, should be complemented with broader evaluations. Organizations stand to gain by emphasizing holistic attributes such as adaptability, critical thinking, and collaboration. The engineering landscape beckons for engineers who can not only code, but also envision, communicate, and innovate.

%------------------------------------------------

\section{Conclusion}

This paper challenges the prevailing narrative that engineers must possess exhaustive knowledge and underscores the limitations of code challenges. It advocates for an inclusive assessment approach that values problem-solving, adaptability, and teamwork. Engineering is an ever-evolving discipline, necessitating engineers who can navigate complexities and contribute meaningfully to the collaborative process \cite{Smith2020,Johnson2019,Brown2021,Garcia2022}.

%----------------------------------------------------------------------------------------
%    REFERENCES
%----------------------------------------------------------------------------------------

\printbibliography % Output the bibliography

%----------------------------------------------------------------------------------------

\end{document}
